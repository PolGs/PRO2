Este documento html documenta la especificación de un codificador/decodificador de mensajes. El programa principal se encuentra en el archivo main.\+cc. A parte de este archivo también són necesarias la siguientes 4 clases\+:~\newline
 -\/{\bfseries \hyperlink{cjt__idioma_8hh}{cjt\+\_\+idioma.\+hh}} que se encarga de gestionar conjuntos de idiomas~\newline
 -\/{\bfseries \hyperlink{idioma_8hh}{idioma.\+hh}} con el que podemos definir un nuevo idioma a partir de un nombre y una tabla de frecuencias.~\newline
 -\/{\bfseries \hyperlink{treecode_8hh}{treecode.\+hh}} que se encarga de traducir las tablas de frecuencias a arbol binario y después codificar/decodificar mensajes recorriendo los árboles.~\newline
 -\/{\bfseries \hyperlink{tabla_8hh}{tabla.\+hh}} se encarga de leer,almacenar y escribir las tablas de frecuencias.~\newline
 